
\documentclass[12pt,a4paper]{report}
% scrartcl ist eine abgeleitete Artikel-Klasse im Koma-Skript
% zur Kontrolle des Umbruchs Klassenoption draft verwenden


% die folgenden Packete erlauben den Gebrauch von Umlauten und �
% in der Latex Datei
\usepackage[utf8]{inputenc}
% \usepackage[latin1]{inputenc} %  Alternativ unter Windows
\usepackage[T1]{fontenc}
\usepackage[ngerman]{babel}
\usepackage{dsfont} %f�r die Einheitsmatrix

\usepackage[pdftex]{graphicx}
\usepackage{latexsym}
\usepackage{amsmath,amssymb,amsthm}

\usepackage{pdfpages} %PDF einbinden

%\usepackage{verbatim} %F�r das Erstellen von kleinen Diagrammen
%\usepackage{tikz}
%\usetikzlibrary{arrows,shapes}

%\usepackage{floatflt} % F�r den float um Diagramme
%\usepackage{float}


\usepackage{geometry} % Bestimmung der Seitenr�nder
\geometry{left=3cm,bottom=3cm,top=3cm,right=3cm} % Definition der Rahmen 

% Die Fussnoten werden durchgehend nummeriert.
\usepackage{chngcntr}
\counterwithout{footnote}{chapter}



\begin{document}



% Umgebungen f�r Definitionen, S�tze, usw.
% Es werden S�tze, Definitionen etc innerhalb einer Section mit
% 1.1, 1.2 etc durchnummeriert, ebenso die Gleichungen mit (1.1), (1.2) ..

\newtheoremstyle{dotless}{}{}{\itshape}{}{\bfseries}{}{ }{}
\theoremstyle{dotless}

\newtheorem{prop}{Proposition}[section]
\newtheorem{defi}[prop]{Definition}
\newtheorem{sat}[prop]{Satz}
\newtheorem{bem}[prop]{Bemerkung}
% \newtheorem*{bem}{Bemerkung}
\newtheorem{lem}[prop]{Lemma}
\newtheorem{kor}[prop]{Korollar}
\newtheorem{thrm}[prop]{Theorem}
\newtheorem{defukor}[prop]{Definition und Korrolar}
\newtheorem{alg}[prop]{Algorithmus}
\newtheorem*{bew}{Beweis}
\newtheorem{bsp}{Beispiel}

\numberwithin{equation}{section}


% einige Abkuerzungen
\newcommand{\C}{\mathbb{C}} % komplexe
\newcommand{\K}{\mathbb{K}} % K�rper
\newcommand{\R}{\mathbb{R}} % reelle
\newcommand{\Q}{\mathbb{Q}} % rationale
\newcommand{\Z}{\mathbb{Z}} % ganze
\newcommand{\N}{\mathbb{N}} % natuerliche
\newcommand{\A}{\mathbb{A}} % Annahme
\newcommand{\OO}{\mathcal{O}} % geschwungenes O f�r Ganzheitsring
\newcommand{\NN}{\mathcal{N}} % geschwungenes N f�r Norm
\newcommand{\IdealeA}{\mathcal{A}}
\newcommand{\IdealeI}{\mathcal{I}}

\newcommand{\ideala}{\mathfrak{a}}
\newcommand{\idealb}{\mathfrak{b}}
\newcommand{\idealc}{\mathfrak{c}}
\newcommand{\idealp}{\mathfrak{p}}
\newcommand{\idealq}{\mathfrak{q}}




  % Keine Seitenzahlen im Vorspann
  \pagestyle{empty}


  % Titelblatt der Arbeit

%\begin{thebibliography}{Schmidt}% In geschweiften Klammern steht ein Text als Platzhalter
%  % Literaturbeispiel: Buch
%  \bibitem[Sch07]{Schmidt} Alexander Schmidt: {\it Einf\"uhrung in die algebraische Zahlentheorie}. %Springer-Verlag, Berlin, Heidelberg, New York, 2007. 
%	\bibitem[Neu]{Neukirch} J\"urgen Neukirch: {\it Algebraische Zahlentheorie}. Springer-Verlag, Berlin, %Heidelberg, New York, 2007
%  % Literaturbeispiel: Paper
%  \bibitem[Anh1]{IntBase} {\it Computing Integral Bases}. Seiten 1 und 4 \\ %http://media.wix.com/ugd/18f425\_58e29a3ba780690f4b00fd9f66a5c529.pdf vergleiche Anhang.
%\end{thebibliography}
%\addcontentsline{toc}{chapter}{Literaturverzeichnis}

\end{document}
